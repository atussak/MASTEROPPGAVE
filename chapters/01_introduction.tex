% !TeX root = ../main.tex
% Add the above to each Chapter to make compiling the PDF easier in some editors.

\chapter{Introduction}\label{Chapter:introduction}

\section{Previous work}

The department of engineering cybernetics at the Norwegian University of Science and Technology (NTNU) has made significant contributions to the field of snake robot control, related to both aquatic snake-like propulsion and efficient snake robot locomotion on flat surfaces \cite{StavdahlNote}.

Some common modes of snake robot locomotion on flat surfaces are lateral undulation, concertina locomotion and sidewinding. Terrestrial snake robot locomotion can also be achieved by utilizing irregularities in the terrain, such as rocks and walls, referred to as obstacles. Directional compliance, proposed by Wang et al. \cite{wang2020directional}, is a method that adapts the snake robot shape to obstacles it collides with in the environment and use them to enhance the existing propulsion mode and avoid getting jammed between obstacles. \textit{Obstacle Aided Locomotion (OAL)} is on the other hand a method that uses obstacles as the main part of the locomotion strategy and always seeks to find the obstacles that can profit the propulsion the most. The idea of this method, which was introduced by Transeth et al. \cite{transeth2008snake}, is that the snake robot pushes against obstacles in order to propel itself forward along a predefined path.

%This method relies on the force sensor feedback received at any point in time in order to calculate the desired shape of the robot. \hl{skriv bedre takk}

In OAL it is desired to control the force against obstacles and shape of the robot independently. Thus, Stavdahl \cite{StavdahlNote} proposed a combination of OAL and \textit{Hybrid Position/Force Control (HPFC)}, leading to the term \textit{Hybrid Obstacle Aided Locomotion (HOAL)}. Klafstad \cite{TorjusOppg} summarized the theory around this concept, and it was further researched and tested by the author \cite{AtussaProsjektoppgp}, both with emphasis on the HPFC method introduced by West and Asada \cite{west1985method} in 1985. This method was developed after the concept was first introduced by Raibert et al. in 1981 \cite{raibert1981hybrid}, and aims at controlling constrained robot manipulators based on kinematic projections of the motion and force such that these variables comply with both the allowable and desired behaviour of the snake robot. Yoshikawa \cite{yoshikawa1987dynamic} advanced the method in 1987 by including the dynamics of the robot in the control calculations. Because the behaviour of the snake robot is driven by its dynamics, the research is now continued to explore the dynamic HPFC method in light of HOAL. 


\section{Scope of the project}

\subsection{Thesis assignment interpretation}

The detailed assignment for this thesis is provided in the very beginning of the report. The author's perception is that the main focus lies on mathematically adapting the dynamic HPFC method of Yoshikawa \cite{yoshikawa1987dynamic} to the snake robot case with special attention to the snake robot force and position spaces. A control structure should then be designed and implemented in a carefully selected simulator to conduct experiments that help evaluate important aspects of the developed method. A discussion of how the dynamic HPFC method can be used to achieve HOAL should also be carried out.
%and the limitations of the HOAL problem in light of which tasks can be realized based on the position and force spaces of the robot.

%This section thoroughly describes the authors interpretation of the different tasks.

%It is understood that some background theory on existing terrestrial locomotion strategies should be provided. It should then be discussed  whether or not they can be combined with the dynamic HPFC method.
%Furthermore, the dynamic HPFC method of Yoshikawa \cite{yoshikawa1987dynamic} is to be mathematically adapted to snake robots. Special attention should here be payed to the modifications that have to be made as a result of the differences between snake robots and traditional robot manipulators. A discussion should be carried out of how the dynamic HPFC method can be used to achieve OAL, and the limitations of the HOAL problem in light of which tasks can be realized based on the position and force spaces of the robot. Furthermore, some sufficient and necessary conditions for propulsion are to be suggested.

%Lastly, a control structure to test the dynamic HPFC algorithm should be designed. A suitable simulator needs to be found and evaluated for this purpose. The performance of the dynamic HPFC method on snake robots should be tested.


\subsection{Simplifications}

Snake robots performing OAL are typically in contact with several obstacles at a time and it is thus desired to control the snake robot at several contact points simultaneously. A higher number of joints and therefore actuators are beneficial for execution of this kind of control. However, in order to reduce the dimensions of the dynamic and kinematic calculations, the snake robot model used for testing consists of six links and five joints, which has limited the possible test variations.

The computed snake robot model is designed for a very specific scenario in which it is assumed that there are three contacts between a predefined set of snake robot links and obstacles at all times. Consequently, all movement of the snake robot can only be performed in the bounds of this assumption, meaning at most one link length displacement.

Another simplification made in the mathematical model is that the snake robot and obstacles are assumed to be 2D bodies, although they have 3D properties in the simulator. However, since all movement and control happens in the 2D plane and all contacts are point contacts, it is not considered to have been a very evident deficiency.

The collisions between the snake robot and obstacles are not modeled because it is assumed that the contact is maintained at all times. However, small collisions do occur in the simulator as a result of the sensitive force sensor and changing contact points within the range of a single link.

Further model assumptions are presented in \ref{seq:assumptions}.

\subsection{Contributions}

The most significant contributions from this project lie in the mathematical adaptation of the dynamic HPFC method to snake robots intended to perform HOAL. An analysis of further control requirements and associated improvements and modifications to the method are also provided. Furthermore, a discussion is carried out of the requirements to the snake robot itself and its configuration relative to the obstacles it is in contact with.

The developed snake robot model and dynamic HPFC method has been tested in a simulation environment after a simulator was thoroughly chosen. Furthermore, a review of this simulator and a comparison to other corresponding simulators is provided. A guiding description of the simulator structure and how it generally can be utilized is also prepared and to be found in the report. Lastly, the findings have been discussed in light of HOAL and improvements to the dynamic HPFC method and control structure are proposed.



%\begin{itemize}
 %   \item Theoretical base
 %   \begin{itemize}
 %       \item A summary of the most significant snake robot locomotion strategies and a discussion of HPFC in light of these locomotion strategies
 %       \item Description of the snake robot specific kinematics and dynamics, with a focus on constrained snake robots
 %   \end{itemize}
 %   
 %   \item Dynamic HPFC for snake robots
 %   \begin{itemize}
 %      \item A thorough mathematical derivation of the dynamic HPFC for snake robots
 %       \item Modifications for increased control performance
 %       \item A discussion of ideal snake robot and obstacle configurations for general dynamic HPFC and for OAL
 %       \item An analysis of the benefits and limitations with special regard to snake robots vs. robot manipulators
  %      \item An analysis of the exploitation of the position and force spaces for propulsion
   %     \item A description of further requirements for the implementation of HOAL and dynamic HPFC on snake robots 
    %\end{itemize}
    %
    %\item Simulations
    %\begin{itemize}
    %    \item A description of the structure of the chosen simulator and comparison to corresponding simulators
    %    \item An evaluation of the validity of the simulators physics engine
    %    \item Implementation of the adapted dynamic HPFC method in the simulator
    %    \item An evaluation of the method through simulation experiments
    %\end{itemize}
%
%\end{itemize}


\section{Report structure}

It is assumed that the reader is familiar with basic robotics and control theory. Please see \cite{lynch2017modern}, \cite{lynch2017modernCompTorque}, \cite{waldron2016kinematics}, \cite{liljeback2012snake} for a more thorough explanation of the topics.

To begin with, the report introduces the snake robot model and related assumptions in Chapter \ref{ch:model_specs}. A set of different terrestrial snake robot locomotion strategies are then presented in Chapter \ref{chapter:theory}. This chapter also covers the required mathematical background theory for understanding the dynamical HPFC method. The dynamic HPFC method is then thoroughly explained in Chapter \ref{ch:hpfc}. All related considerations and limitations are also provided here. Chapter \ref{chapter:simulator} explains the structure of the chosen simulator and how it is used for the experiments in Chapter \ref{ch:results}. The results from the experiments and general analysis of the dynamic HPFC method on snake robots are discussed in Chapter \ref{ch:discussion}. Lastly, Chapter \ref{ch:conclusion} concludes the work and proposes some ideas for future research and possible improvements based on the challenges encountered in this project.