% !TeX root = ../main.tex
% Add the above to each Chapter to make compiling the PDF easier in some editors.

\chapter{Introduction}\label{Chapter:introduction}

\section{Previous work}

The department of engineering cybernetics at the Norwegian University of Science and Technology (NTNU) has made significant contributions to the field of snake robot control, related to both aquatic snake-like propulsion and efficient snake robot locomotion on flat surfaces \cite{StavdahlNote}.

Some common modes of snake robot locomotion on flat surfaces are lateral undulation, concertina locomotion and sidewinding. Terrestrial snake robot locomotion can also be achieved by utilizing irregularities in the terrain, such as rocks and walls, referred to as obstacles. Directional compliance, proposed by Wang et al. \cite{wang2020directional}, is a method that adapts the snake robot shape to obstacles in the environment and use them to enhance the existing propulsion mode. Obstacles can also be utilized for propulsion without any underlying locomotion strategy. A central method here is \textit{Obstacle Aided Locomotion (OAL)} introduced by Transeth et al. \cite{transeth2008snake}, in which snake robots push against obstacles in order to propel themselves forward along a predefined path.

%This method relies on the force sensor feedback received at any point in time in order to calculate the desired shape of the robot. \hl{skriv bedre takk}

In OAL it is desired to control the force against obstacles and shape of the robot independently. Thus, Stavdahl \cite{StavdahlNote} proposed a combination of OAL and \textit{Hybrid Position/Force Control (HPFC)}, leading to the term \textit{Hybrid Obstacle Aided Locomotion (HOAL)}. Klafstad \cite{TorjusOppg} summarized the theory around this concept, and it was further researched and tested by the author \cite{AtussaProsjektoppgp}, both with emphasis on the HPFC method introduced by West and Asada \cite{west1985method} in 1985. This method was developed after the concept was first introduced by Raibert et al. in 1981 \cite{raibert1981hybrid}. The method aims at controlling constrained robot manipulators based on kinematic projections of the motion and force such that these variables comply with both the allowable and desired behaviour of the snake robot. Yoshikawa \cite{yoshikawa1987dynamic} advanced the method in 1987 by including the dynamics of the robot into the control calculations. Because the behaviour of the snake robot is driven by its dynamics, the research is now continued to explore the dynamic HPFC in light of HOAL. 


\section{Scope of the project}

\subsection{Thesis assignment interpretation}

The assignment for this thesis is provided in the very beginning of the report. This section thoroughly describes the authors interpretation of the different tasks.

It is understood that some background theory on existing terrestrial locomotion strategies should be provided. It should then be discussed  whether or not they can be combined with the dynamic HPFC method.
Furthermore, the dynamic HPFC method of Yoshikawa \cite{yoshikawa1987dynamic} is to be mathematically adapted to snake robots. Special attention should here be payed to the modifications that have to be made as a result of the differences between snake robots and traditional robot manipulators. A discussion should be carried out of how the dynamic HPFC method can be used to achieve OAL, and the limitations of the HOAL problem in light of which tasks can be realized based on the position and force spaces of the robot. Furthermore, some sufficient and necessary conditions for propulsion are to be suggested.

Lastly, a control structure to test the dynamic HPFC algorithm should be designed. A suitable simulator needs to be found and evaluated for this purpose. The performance of the dynamic HPFC method on snake robots should be tested.


\subsection{Simplifications}

In order to keep the dimensions of the dynamic and kinematic calculations down, the snake robot model used for testing consists of six links and five joints. Because snake robots performing OAL are in contact with several obstacles, it is desired to control several contact points as well as the shape of the robot. This means that a higher number of joints and thus actuators are beneficial for the execution of the control. Unfortunately, this led to a lack of test variations.

The computed snake robot model is designed for a very specific scenario in which three contacts between snake robot links and obstacles have to be maintained. Consequently, all movement of the snake robot can only be performed in the bounds of this assumption. 

Another simplification made in the mathematical model is considering the snake robot and obstacles as 2D bodies, although they have 3D properties in the simulator. However, since all movement and control happened in the 2D plane and all contacts are still point contacts, it is not considered to have been a very big deficiency.

The collisions between the snake robot and obstacles are not modeled because it is assumed that the contact is maintained at all times. However, small collisions do occur in the simulator as a result of the sensitive force sensor and changing contact points within the range of a single link.

\subsection{Contributions}

\begin{itemize}
    \item Theoretical base
    \begin{itemize}
        \item A summary of the most significant snake robot locomotion strategies and a discussion of HPFC in light of these locomotion strategies
        \item Description of the snake robot specific kinematics and dynamics, with a focus on constrained snake robots
    \end{itemize}
    
    \item Dynamic HPFC for snake robots
    \begin{itemize}
        \item A thorough mathematical derivation of the dynamic HPFC for snake robots
        \item Modifications for increased control performance
        \item A discussion of ideal snake robot and obstacle configurations for general dynamic HPFC and for OAL
        \item An analysis of the benefits and limitations with special regard to snake robots vs. robot manipulators
        \item An analysis of the exploitation of the position and force spaces for propulsion
        \item A description of further requirements for the implementation of OAL and dynamic HPFC on snake robots 
    \end{itemize}
    
    \item Simulations
    \begin{itemize}
        \item A description of the structure of the chosen simulator and comparison to other comparable simulators
        \item An evaluation of the validity of the simulators physics engine
        \item Implementation of the dynamic HPFC in the simulator
        \item An evaluation of the dynamic HPFC through simulation experiments
    \end{itemize}

\end{itemize}


\section{Report structure}

It is assumed that the reader is familiar with basic robotics and control theory. Please see \cite{lynch2017modern}, \cite{lynch2017modernCompTorque}, \cite{waldron2016kinematics}, \cite{liljeback2012snake} for a more thorough explanation of the topics.

The report first introduces the snake robot model and related assumptions employed in the project in Chapter \ref{ch:model_specs}. A set of different terrestrial snake robot locomotion strategies are then presented in Chapter \ref{chapter:theory}. This chapter also covers the required mathematical background theory to understand the dynamical HPFC method. The dynamic HPFC method is then thoroughly explained in Chapter \ref{ch:hpfc}. All the related considerations and limitations are also provided here. Chapter \ref{chapter:simulator} explains the structure of the chosen simulator and how it is used for the experiments in Chapter \ref{ch:results}. The results from the experiments and general analysis of the dynamic HPFC method on snake robots are discussed in Chapter \ref{ch:discussion}. Lastly, Chapter \ref{ch:conclusion} concludes the work and proposes some ideas for future research and possible improvements based on the challenges encountered in this project.