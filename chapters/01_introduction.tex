% !TeX root = ../main.tex
% Add the above to each Chapter to make compiling the PDF easier in some editors.

\chapter{Introduction}\label{Chapter:introduction}

\section{Previous work}
- NTNU and snake robots
- Terrestrial snake robot locomotion (common approaches)
- OAL
- The need to control both position and force led to the familiarization with HPFC.
- HPFC Raibert+Craig and West+Asada
- Stavdahl and the term HOAL
- Summarized by Klafstad
- Concept tested by the author in previous project
- Because the behaviour of a snake robot is "driven" by its dynamics, the research is now continued to explore the dynamic HPFC for snake robots in light of OAL

\section{Scope of the project}

\subsection{Thesis assignment interpretation}

- Introduce some existing locomotion strategies and discuss whether or not they could be combined with dynamic HPFC.
- Mathematically adapt the method of Yoshikawa to the snake robot, paying special attention to the changes that have to be made as a result of the differences between snake robots and traditional robot manipulators.
- Discuss how the dynamic HPFC method could be used to achieve OAL.
- Discuss the limitations of the HOAL problem in the light of which tasks (in force and position space) can actually be realized.
- Discuss what force/position spaces are sufficient for propulsion and which are necessary for propulsion in light of the task realizations(?).
- Design a control structure to test the dynamic HPFC algorithm.
- Find and evaluate a simulator for the testing
- Test the performance of the dynamic HPFC method on snake robots. The tests should shed light on the benefits and limitations of the method.

\subsection{Simplifications}

- Not very many links in the snake robot. -> led to a a lack of test variations
- Model for a very specific scenario (always contact with three obstacles with the same three links, thus movement can only happen in the bounds of this)
- All links are modeled as 2D lines although they both have a width and are 3D in the simulator.
- Dynamics are generally linearized
- No modeling of collisions although they unwantedly occur as a result of the super hyper sensitive force sensors in the simulator and the physiscs engine constantly trying to push the bodies apart.

\subsection{Contributions}

\hl{Consider making points and sub-points here}

\begin{itemize}
    \item A summary of the most significant snake robot locomotion strategies and a discussion of HPFC in light of these locomotion strategies
    \item Thorough mathematical derivation of the dynamic hybrid position force control of snake robots
    \item Analysis of the control structure benefits and limitations
    \item Analysis of the snake robot obstacle aided locomotion scheme with regard to the environment and snake robot configuration
    \item Discussion/explanation of the differences with respect to traditional robot manipulators
    \item Description of the simulator structure and comparison to other comparable simulators
    \item An evaluation of the validity of the Gazebo/ROS simulator
    \item Implementation of the controller in the Gazebo/ROS simulator.
    \item An evaluation of the control scheme through simulation experiments.
    \item A comprehensive description of further requirements for the implementation of OAL and dynamic HPFC on snake robots 
\end{itemize}


\section{Report structure}