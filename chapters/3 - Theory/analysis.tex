\section{Constraint space analysis}

\section{Task analysis}

Even though the control solution presented in \ref{subsec:DHPFC} decomposes the workspace in force and movement directions, there are some restrictions as to which tasks can be performed by the robot. To analyze this, it is beneficial to look at what kind of tasks a snake robot is required to achieve. In particular, the tasks of a snake robot moving according to the OAL principle will be covered in this section. Both a higher level path following goal and a lower level control goal is presented. It is the lower level goal that will be further researched in this project. However, it is important to keep in mind what the purpose of the snake robot eventually is.

Lastly, the task restrictions are discussed, followed by a mathematical adaption of the control scheme in \ref{subsec:DHPFC}. This adaption is constructed to allow for the control of the task-relevant variables in cases where all variables can not be controlled independently.

\subsection{The overall task of the snake robot}

The motivation behind implementing the OAL principle on a snake robot is to make it move from one point in space to another using the obstacles present in its environment. In addition to this, it is typical that snake robots moving in environments cluttered with obstacles are set to document their surroundings with a camera or similar equipment. This again makes their exact path from the starting point to the end point relevant. Thus, the path following of the head of the snake robot is the overall goal for the snake robots focused on in this project. When a path has been designed, the overall goal can be decomposed in tasks consisting of the global position and orientation of the snake robot head. In traditional robot manipulator theory, this would be referred to as the end-effector movement. 


\subsection{Lower level control tasks}

In order to reach the higher level path following goal, the snake robot has to push itself forward in a purposeful manner utilizing the obstacles in the environment. By purposeful it is here meant that the direction of the force application against the obstacles have to conform with the desired propulsion direction given by the path. This is where the hybrid position force control comes into play. The robot has to both position itself in a certain manner alongside the obstacles and push against them with a given force magnitude.

The general idea is that at any point in time a task will be given by a higher level path following algorithm which is from now assumed implemented. The information provided should include which obstacles are to be utilized and how the snake robot is to utilize them. In other words, the tasks sent to the hybrid position force controller simply consist of a desired positioning (orientation) by every obstacle and force to be applied to the obstacles.



\subsection{Task restrictions}

The question that now arises is what the restrictions are to which tasks the higher level path following algorithm can command the snake robot to perform. It is obvious that not any given combination of position and force can be simultaneously achieved.

The restrictions lie in the composition of the snake robot, meaning how many joints the robot consists of. A higher number of joints, and thus actuators, enables the robot to control a higher number of variables. At the same time, a higher number of contact points impose a higher number of constraints on the system and therefore limit the controllability.

For every contact point there are three possible variables that could be controlled. In \ref{subsec:DHPFC} it is defined that two of these variables belong to the position control. That is the movement along the obstacle and rotation by or around the obstacle. The remaining variable is force against the obstacle.

To analyze the restriction to how many and which of these variables can be controlled for a given robot it is useful to look at the closed kinematic chains in the robot. A closed kinematic chain is \hl{blalblablallablbal}. In the snake robot there is one closed kinematic chain from the base of the robot to the end effector, one from the base of the robot to every obstacles and one between every set of obstacles. To better explain this an example is visualized in Figure \ref{}. \hl{figur av masse CKCs.}

The actuators possible to use for control of the variables at a single contact point are limited by the preceding contact point. This is because the closed kinematic chain between these two points always will be the smallest one with the least number of actuators and thus the so called bottleneck for the control. To better explain this, the example in Figure \ref{} is again treated. Joints 4 and 5 are available for controlling variables at contract point 3.
If the angle of link 4 is arbitrary, joint 3 can be employed as well. If however, this angle has to be controlled to follow a position or force setpoint for contact point 2, the joint is "occupied" and the total number of actuators available is two. That means that for instance a combination of force application against and rotation about obstacle 3 can be achieved.

The ideal way of designing snake robots that require a high level of controllability is to include a very high number of joints, making it hyper redundant. \hl{hyperredundant robots}.


\subsection{Task oriented control scheme}


In \ref{subseq:HPFC} it is explained that there are certain directions of force and movement that are essential for performing a task. It is further claimed that the number of essential directions can not be greater than the number of controllable actuators in the robot necessary for performing the task. This is an important remark in the case of snake robots with passive joints, as none of these joints are controllable.

For the snake robot the force and movement directions of a task are described by $\mathbf{E}_F$ and $\mathbf{E}_P$. In some cases, the number of directions described by these vectors might exceed the number of active joints even though not all directions have to be controlled precisely in order to perform the task. So in other words, the number of essential directions in $\mathbf{E}_F$ and $\mathbf{E}_P$ together can not exceed the number of active joints in the robot. An example is a robot with 8 active joints in contact with 4 obstacles. The total number of force and movement directions described by its $\mathbf{E}_F$ and $\mathbf{E}_P$ is 3 per contact point, meaning 12 directions. This is obviously greater than 8 and the force and position can not be controlled independently in all these directions.

Another very important factor to note here is that this logic is described for a closed kinematic chain. This is in the snake robot case the joints connecting the virtual base to the contact point. Because of this, active joints \textit{after} the contact point do not contribute to greater controllability. Furthermore, the segments between the contact points can be seen as closed kinematic chains. For this reason, the placement of the obstacles by the snake has an impact on how many directions can be controlled independently. Thus, the number of directions that can be controlled for the mentioned example might not be 8. If the 4 obstacles are placed by the first 4 links, only three directions can be controlled independently. \hl{These have to be position directions??? Research further!!}

The essential position and force directions can be described by (\ref{eq:hpfc_wep}) and (\ref{eq:hpfc_wef}). The filters (\ref{eq:hpfc_fp}) and (\ref{eq:hpfc_ff}) can then be used to focus the control on these essential directions. This is of course given that the requirements regarding the number of active joints are fulfilled. The new input control torques can now be found by

\begin{equation}
    \boldsymbol{\tau}_c = \prescript{}{j}{\mathbf{F}}_{p} \boldsymbol{\tau}_P + \prescript{}{j}{\mathbf{F}}_{f} \boldsymbol{\tau}_F
\end{equation}
\\
\hl{But can Fp be used like this thoooo?? If yes, write why cuz it is an approximation}
