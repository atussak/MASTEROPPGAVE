\section{Snake robot constraint formulation}\label{seq:constraints}

Figure \ref{fig:obst-coords} illustrates the used model of the snake robot and its environment. When looking at this model, it is evident that the snake robot will experience constraints if it tries to move its links in the directions of the obstacles. One way of expressing this constraint is limiting the velocity of the links in contact. The position of contact on link $i$ was in \ref{subseq:constr_inst} denoted as $(x_{c,i}, y_{c,i})$. Thus, the corresponding velocity $\mathbf{v}_{c,i}$ can be written $(\dot{x}_{c,i}, \dot{y}_{c,i})$.

The velocity of this point is not constrained in every direction, but in the direction normal to the link. This follows from the assumption of the contact being a point contact. The normal velocity can be found with the use of the angle $\theta_i$ of link $i$ with respect to the base frame. From basic trigonometric laws, the normal velocity is found to be

\begin{equation}\label{eq:norm_vel}
    v_{nc,i} = -sin(\theta_i) \dot{x}_{c,i} + cos(\theta_i) \dot{y}_{c,i}.
\end{equation}
\\
Since there can only be one contact per link and the contact can only take place at one side of the link at the time, the link can still move away from the obstacle with a velocity normal to the link. Holden et al. \cite{holden2014optimal} introduce a variable $\gamma_i$ that holds information about which side of the link the obstacle is positioned. The variable can take the values ${-1, 0, 1}$, where $\gamma_i=1$ for obstacles to the right of the link and vice versa. $\gamma_i=0$ for links not in contact with any obstacles.

The allowed normal direction of movement for link $i$ can now be expressed as

\begin{equation}\label{eq:norm_vel2}
    \gamma_i v_{nc,i} \geq 0.
\end{equation}
\\
It might be desired to stay in touch with the obstacle for propulsion purposes. Because of this, it is reasonable to simplify the constraint equation to not let the contact point on the link move in any normal direction to the link at that point. The constraint can thus be expressed as

\begin{equation}\label{eq:norm_vel3}
    \gamma_i v_{nc,i} = \gamma_i(-sin(\theta_i) \dot{x}_{c,i} + cos(\theta_i) \dot{y}_{c,i}) = 0
\end{equation}
\\
Moreover, the side of the contact is insignificant if the velocity is limited in both directions. (\ref{eq:norm_vel3}) is further simplified to

\begin{equation}\label{eq:norm_vel4}
    v_{nc,i} = -sin(\theta_i) \dot{x}_{c,i} + cos(\theta_i) \dot{y}_{c,i} = 0.
\end{equation}
\\
In robot manipulator theory it is common to use both the base coordinate frame and a coordinate frame related to the end effector. The end effector coordinate system is irrelevant for the contact point velocities. It is however possible to define the velocities in coordinate systems related to the contact points. A coordinate system related to the contact point on link $i$ is $(x_{t,i}, y_{t,i})$, where the x-direction is the direction of the link in contact. This means that the rotation of this coordinate system relative to the base coordinate system is $\theta_i$. The subscript $t$ stands for \textit{task} and is introduced for its purpose in the dynamic hybrid position controller explained in \ref{subsec:DHPFC}.

Since the velocity normal to the link in the task coordinate frame is the velocity along the y-axis, the constraint on link $i$ can be rewritten to

\begin{equation}\label{eq:norm_vel5}
    \dot{y}_{t,i} = 0,
\end{equation}
\\
which is much simpler than (\ref{eq:norm_vel4}).