\section{Background info}

%- Why was this particular simulator chosen (mention MATLAB)?

SnakeSIM is a virtual rapid-prototyping framework made by Sanfilippo et al. \cite{sanfilippo2018snakesim} to allow researchers for the design and simulation of \textit{perception-driven obstacle-aided locomotion} (POAL) in a safe and rapid manner. The core of the simulator is based in the Robotic Operating System (ROS) \cite{quigley2009ros}. This is where the robot model and control is defined. Furthermore, it is connected to Gazebo \cite{koenig2004design}, providing both a robust physics engine and a real-time graphical user interface. SnakeSIM has an additional interface towards the RViz visualization tool. This part will however not be explained further as it has not been applied in this project. Lastly it should be noted that ROS is currently only supported on the Ubuntu operating system, and has a command line based interface.

The simulator has been modified to fit even better to the project. The main contribution is the implementation of the dynamic controller. This is explained further in \ref{sec:4-contributions}. Furthermore, certain properties of the simulator, like visual perception and 3D movement, have been neglected.
The main purpose of the experimentation in the simulator is to compare the performance of the dynamic hybrid position/force controller and other traditional controllers. 

\subsection{Motivation for the simulator choice}

The customization of SnakeSIM towards the case of obstacle-aided locomotion is a crucial reason for why this particular simulator was chosen to conduct the experiments. It offers a setup with the desired frictionless environment, a simple snake robot and obstacles. A simpler version of a simulator with the same purpose was developed in MATLAB by the author \cite{AtussaProsjektoppgp}. However, this simulator lacked the integration of the physics describing the result of interaction between obstacles and the snake robot. The Gazebo physics engine in the SnakeSIM contributes to more realistic results in the experiments compared to what could be achieved with the MATLAB simulator.

Another MATLAB simulator tested is one developed by Transeth et al. \cite{transeth2008snake}. This simulator is also very adapted to the simulation of OAL. Nevertheless, its graphical user interface can not be compared to that of Gazebo. In addition, it is understood that the architecture of SnakeSIM is more modular, which allows for a smoother integration of additions to the program. This is again essentially due to the way ROS is designed. A more thorough explanation of ROS is provided in \ref{sec:sim_architecture}.