\section{Simulator architecture}

The SnakeSIM simulator is based in and run from the Robot Operating System (ROS) \cite{quigley2009ros}. It is a set of open source software libraries and tools designed to build robot applications. It further provides features like message-passing and package management. The ROS-based program is set up to launch the Gazebo 3D simulator, which is adopted to accurately simulate the snake robot in the desired environment with obstacles. All simulated objects making out the snake robot and the obstacles have attributes like mass and velocity, allowing for realistic behaviour when interaction and collisions occur.
It is also possible to interact directly with the snake robot and its environment through the visual presentation provided by Gazebo.

Gazebo is responsible for reporting all physical variables and how they change with the movement within the world. It is also possible to add different kinds of sensors to the robot to retrieve forces, torques, contact positions, etc. The controllers implemented in the ROS platform listen to this data and use it to compute desired motor inputs for the actuators positioned in the snake robot joints. This is again data that Gazebo listens to in order to display the correct movement of the snake robot. The information flow is visualized in Figure \ref{}.

Moreover, the ROS part of the simulator is divided into several so called nodes that can run independently of each other. Some examples are the Gazebo node, the position controller, the dynamic controller, the environment setup configuration, etc. The relevant nodes and the information flow between them are shown in Figure \ref{}. The use of topics for message passing allows for the nodes to send and receive information independently of each other. The data is exchanged by publishing and subscribing to these designated topics (message passing). This means that any node does not have to wait for another node to read or receive the published data before continuing with its computations.

The configuration of the simulated snake robot and obstacles is implemented according to the Universal Robotic Description Format (URDF) \cite{}. This is a file format used to describe elements like links, joints, sensors and actuators and how these elements connect to each other.