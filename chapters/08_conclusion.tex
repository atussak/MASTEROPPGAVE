\chapter{Conclusion}\label{ch:conclusion}

This project has researched the application of dynamic HPFC on snake robots. The focus has been on controlling variables that are believed necessary for achieving HOAL. That includes the position and orientation of the snake robot alongside obstacles and the contact force between the snake robot and obstacles. A lot of the contributions to the research topic has been formulating definitions for the snake robot dynamic HPFC problem.

From the experiments it can be concluded that just the modifications made to the dynamic HPFC method are not sufficient for controlling the snake robots several contact points, especially for snake robots with a moderate amount of joints. However, the performance was considerably increased with the additions for restricting which joints should be commanded for which controls.

The control managed to steer the position and force to follow constant reference values within small bounds of the initial configuration of the snake robot. The low number of snake robot joints limited the experiments and only very specific scenarios could be tested. It can be concluded that the number of snake robot links relative to the number of obstacles is very important for controllability. More specifically, it is important how many links are between every contact with an obstacle. This was researched through the study of the different closed kinematic chains present in the snake robot and the inclusion of these in the control.

One solution for regarding the presence of passive unactuated joints was both presented and tested. From the experiment it is obvious that some adaption of the dynamical joint torque calculation is necessary, although this method might not be the most ideal. The solution should be studied and developed further and adapted more thoroughly to the snake robot case.

Although mastering the ROS and Gazebo frameworks is not very elementary, the SnakeSIM simulator is a great platform for studying HOAL. The most significant deficiency for this project has been the force sensor on the simulated snake robot, which can send quite unsteady signals if the snake robot is moving ever so slightly while in contact with an obstacle. It also has some threshold for when contact is registered by the sensor although the snake robot does seem to be in contact with obstacles judging from the visual part of the simulation.

Surely, a great deal of further testing is required to support the developed dynamic HPFC method for snake robots. There is also room for improvements, both when it comes to the mathematical foundation and the test environment. These aspects are regarded in the following section.

\section{Future work}

As mentioned earlier, more testing should be carried out to evaluate the developed method for dynamic HPFC of snake robots. Since it has been understood that snake robots with a large number of links are ideal for the HOAL scheme, it is desired to conduct further testing with a much bigger snake robot. The tests should still be conducted in an enclosed simulator test environment so that different aspects can be analyzed under isolation of unknown outer disturbances and with accurate position/force data. The SnakeSIM simulator is recommended for further tests as well. However, a better method of calculating the dynamics of a snake robot with arbitrarily many links should be implemented.

Furthermore, with a higher number of links a greater amount of solutions to the control problem will presents themselves. Thus it is desired to analyze the arbitrariness of the behaviour of hyperredundant snake robots. Special emphasis should be put on addressing the arbitrariness term in the calculation of the desired joint accelerations.

A better solution for isolating the commanded torques to active joints should also be researched. By preference, the proposed solution can be adopted, but it is recommended to find a more mathematical and automatic way of finding which joints should be controlled precisely and which can take arbitrary values as a result of the controlled joints. An idea is looking at the span of the propulsion space and which joints need to be controlled in order to keep the snake robot within this space so that the necessary condition for propulsion is met.

When the dynamic HPFC method for snake robots has been established, it can be combined with the suggested HOAL algorithm. There are of course several other parts that need to be researched and developed as well. This includes the development of a path planner and path following algorithm.

Some ideas of how this can be done............