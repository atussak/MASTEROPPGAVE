\chapter{Conclusion}\label{ch:conclusion}

This project has researched the adaptation and application of dynamic HPFC on snake robots to allow for both natural and virtual constraints on position and force to be met during propulsion. The focus has been on controlling variables that are considered necessary for achieving HOAL. That includes the position and orientation of the snake robot alongside obstacles and the contact force between the snake robot and obstacles. A lot of the contributions to the research topic have been formulations and definitions for the snake robot dynamic HPFC problem.

From the experiments it can be concluded that the modifications made to the dynamic HPFC method are not alone sufficient for controlling the several contact points of any snake robot independently. This is especially the case for snake robots with a moderate number of joints. However, the performance was considerably increased with the developed additions that assign certain actuated joints to certain contacts so that the controls do not overlap.

%restrict which joints are commanded by the control of the different contacts.

The control managed to steer the position and force to follow constant reference values within small bounds of the initial configuration of the snake robot. The low number of snake robot joints limited the experiments and only very specific scenarios could be tested. It can be concluded that the number of snake robot links relative to the number of obstacles is very important for controllability. More specifically, it is important how many links are between every  obstacle contact. This was researched through the study of the different closed kinematic chains present in the snake robot and the inclusion of these in the control.

One solution for regarding the presence of passive unactuated joints based on the dynamic coupling of the snake robot was presented and tested. From the experiment it is obvious that some adaption of the dynamical joint torque calculation is necessary, although this method might not be the most ideal, as discussed in Chapter \ref{ch:discussion}. The solution should be studied and developed further and adapted more thoroughly to the snake robot case.

Although mastering the ROS and Gazebo frameworks is not very elementary, the SnakeSIM simulator is a great platform for studying HOAL. The most significant deficiency for this project has been the force sensor on the simulated snake robot, which can send quite unsteady signals if the snake robot is moving ever so slightly while in contact with an obstacle. It also has some threshold for when contact is registered by the sensor, meaning that when the snake robot does seem to be in contact with obstacles judging from the visual part of the simulation, the force sensor might not react.

Surely, a great deal of further testing is required to support the developed dynamic HPFC method for snake robots. There is also room for improvements, both when it comes to the mathematical foundation and the test environment. These aspects are regarded in the following section.

\section{Future work}

As mentioned earlier, further testing should be carried out to evaluate the developed method for dynamic HPFC of snake robots. Since it has been understood that snake robots with a large number of links are ideal for the HOAL scheme, it is desired to conduct further testing with a much bigger snake robot. The tests should still be conducted in an enclosed simulator test environment so that different aspects can be analyzed under isolation of unknown outer disturbances and with accurate position and force data available. The SnakeSIM simulator is recommended for further tests as well. However, a better method of calculating the dynamics of a snake robot with arbitrarily many links should be implemented.

Furthermore, with a higher number of links, and thus an increased dexterity level, a greater amount of solutions to the control problem will present themselves. It is therefore desired to analyze the arbitrariness of the behaviour of hyperredundant snake robots. Special emphasis should be put on addressing the arbitrariness term in the calculation of the desired joint accelerations.

A better solution for isolating the commanded control torques to only apply to actuated joints should also be researched. By preference, the proposed solution can be adopted, but it is recommended to identify a more mathematical and automatic way of finding exactly which joints should be controlled precisely and which can take arbitrary values resulting from the controlled joints. A proposition is looking at the span of the snake robot propulsion space and the desired velocity directions along the given path. Seeing as a necessary condition for propulsion is that the velocity direction along the path lies within the propulsion space, the underlying criteria for this could be analyzed and used to study the influence and importance of the different joints, both active and passive.

%which joints need to be controlled in order to keep the snake robot within this space so that the necessary condition for propulsion is met.

When the dynamic HPFC method for snake robots has been established, it can be combined with the suggested HOAL algorithm. There are of course several other parts that need to be researched and established as well. This includes the development of automatic path planner and path following algorithms. These should be based on the criteria for the operation of the snake robot, like minimizing energy consumption (discussed by Holden et al. \cite{holden2013optimal}, \cite{holden2014optimal}) and the traversed distance. Methods like model predictive control (MPC) and reinforcement learning (RL) are suggested to be investigated in future work. Furthermore, the inclusion of the position of the snake robot head to the dynamic HPFC task space vector $\mathbf{r}_t$ could benefit the path following component.

The research field of snake robot OAL/HOAL is growing, and it is therefore believed that a common robust simulation platform for testing would be of great convenience. SnakeSIM is a great base here, but comprehensive work should be put into generalizing it for a more seamless adaptation to different simulation scenarios. A detailed user guide and documentation of the platform would also vastly benefit the research community.