\section{Force control with various CKC formulations}\label{sec:2xminiJforce}

The purpose of this experiment is illustrating the differences between the use of the two closed kinematic chain (CKC) formulations described in \ref{subsec:task-restrictions}. The first simulation is run with minimal CKCs, and the second one is run with the regular CKCs defined with respect to the base frame.

\begin{table}[h!]
    \centering
    \begin{tabular}{|c|c|c|}
        \hline
        & \textbf{Value} & \textbf{Unit}\\
        \hline \hline
        Number of obstacles & $3$ & \\
        Number of links & $6$ & \\
        $f_{F,d}$ & $2$ & $N$ \\
        $[K_{p}, K_{i}]$ & $[0.5, 0.003]$ &\\
        \hline
    \end{tabular}
    \caption{Simulation configuration for force control experiment}
    \label{tab:exp_2xf}
\end{table}

The snake robot is in contact with three obstacles, whereas the force against the second and third obstacle is controlled. The desired force magnitude $f_{F,d}$ is 2 $N$ for both contacts. The simulation configuration is summarized in Table \ref{tab:exp_2xf} and is common for both simulations. The control structure for this experiment is given in Figure \ref{fig:diag-f}.



\begin{figure}[h!]
    \centering
    \includegraphics[trim=0cm 0cm 3cm 0cm, clip=true, width=\textwidth]{figures/experiments/control-diagrams/2f-control-diagram.pdf}
    \caption{Control diagram for force control}
    \label{fig:diag-f}
\end{figure}

It should be mentioned that very thorough tuning of control parameters could improve the results of both simulations, although a great deal of tuning already has been conducted. Furthermore, the control is quite twitchy as a result of the unsteady force sensor feedback. The experiment in \ref{sec:pos-force-control-exp} is conducted with a low pass filtering of these sensor signals, and proves that the control consequently gets smoother as well.

\newpage
\subsection{Regular CKCs}

For the first simulation, the regular CKC formulations are used. This means that all contact points are described with respect to the base frame of the robot, as illustrated in Figure \ref{fig:CKC1} in \ref{subsec:task-restrictions}. It can be seen that the two CKCs for the second and third obstacle are overlapping. As a result, the first two joint motors will be used to control both contact points. The motors for joints 3 and 4 will however still be reserved to the control of the third contact point. From the input torque plot in Figure \ref{fig:2xf-bigJ} it can be observed that all joint motors except for the last one are actuated. The last one is left out as it is positioned after both controlled contact points. 

Figure \ref{fig:2xf-bigJ} also presents the resulting contact forces. As expected, the contact force for the third contact point is much better controlled than for the second contact point. This is because it has more actuators available, again making it more robust. In addition, it is known that the actuators used for the second contact point are shared for both controls. Consequently, the input torques $\tau_3$ and $\tau_4$ used only for the third contact are much more stable than the shared input torques $\tau_1$ and $\tau_2$.

Another reason for why $f_{F,2}$ never reaches its desired value can be that the joint torques related to this control reach the saturation limit, which has an absolute value of 2 in this experiment. Unfortunately, it was found that larger motor torques easily led to the contact being lost at some moments. This is highly undesired as the mathematical model assumes contact at all times.

\begin{figure}[H]
    \centering
    \includegraphics[trim=2cm 2cm 2cm 2cm, clip=true, width=\textwidth]{figures/experiments/2xf/bigJ-2plot.pdf}
    \caption{Results from experiment with regular CKCs}
    \label{fig:2xf-bigJ}
\end{figure}

\subsection{Minimal CKCs}

For the second simulation, the minimal CKC formulations are used, meaning that the CKC for the second obstacle is defined from the first to the second obstacle point, and the CKC for the third obstacle is defined from the second to the third obstacle. This formulation is easier understood from Figure \ref{fig:CKC2} in \ref{subsec:task-restrictions}. It can be observed that the CKC belonging to the third obstacle contains two joints, whereas the second one only contains one joint.

Figure \ref{fig:2xf-miniJ} shows the resulting contact forces and joint torques from the simulation. It can be observed that both forces are close to the desired value. Nonetheless, the force $f_{F,3}$ against the third obstacle is considerably more stable than the force $f_{F,2}$ against the second obstacle. It is not implausible that this is a result of the third contact point having one more joint available for control. This increases the robustness of the control. In addition, it is known that the joint connecting link 3 and 4 will influence the third link, which is in contact with the second obstacle. This can further be seen as a disturbance on the control of the second contact force.

From the control torques in Figure \ref{fig:2xf-miniJ}, it is evident that only the motors on joints 2, 3 and 4 are used. This is logical, as it is exactly what the corresponding CKCs allow. Since the controllability increases with the current minimal CKC formulation, the control is also significantly smoother than in the previous simulation.

\begin{figure}[H]
    \centering
    \includegraphics[trim=1.5cm 2cm 2cm 2cm, clip=true, width=\textwidth]{figures/experiments/2xf/miniJ-2plot.pdf}
    \caption{Results from experiment with minimal CKCs}
    \label{fig:2xf-miniJ}
\end{figure}


%Filtering the sensor signals with a low pass filter could have improved the smoothness of the results. However, the main sense of the experiment is still communicated with the presented results.

