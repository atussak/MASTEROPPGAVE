\chapter{Discussion}\label{ch:discussion}

This chapter looks at the effects of the different modifications and adaptions made to the dynamic HPFC method. The discussion is based on the performed experiments in Chapter \ref{ch:results} and the mathematical analysis carried out in Chapter \ref{ch:hpfc}. It further evaluates the chosen simulator, SnakeSIM, and the implementation of the dynamic HPFC method in this simulator. Lastly, the application and requirements of the dynamic HPFC method for HOAL are assessed. Some improvements to the mentioned topics are also proposed throughout the chapter.

\section{The snake robot dynamic HPFC method}

This section discusses the modifications made to the dynamic HPFC method in order to adapt it to snake robots.

\subsection{Differences with traditional HPFC}\label{subsec:dis-diff}

As the name implies the dynamic HPFC method of Yoshikawa \cite{yoshikawa1987dynamic} described in \ref{subsec:DHPFC} depends on the dynamical model of the robot. This is the main difference from the plain HPFC method of West and Asada \cite{west1985method} described in \ref{subseq:HPFC}, and makes the calculations significantly more complex. It is however believed that it pays off in the resulting ability to predict the dynamical behaviour of the snake robot.

In this project, the dynamics are calculated using the Euler Lagrange method. It was discovered that the computations get quite extensive for snake robots with a large amount of links. An alternative method is provided by Liljebäck et al. \cite{liljeback2012snake}, which is significantly easier to apply to snake robots with many links. The disadvantage with this method is however that it is based on the center of mass of the snake robot, which is an irrelevant parameter for the HOAL problem. %where the main focus is guiding the head of the snake robot.
An option is adapting this method to depend on the position of the tail instead, which is already part of the generalized coordinates in the presented model.

The assumption of no friction made in this project surely simplifies the snake robot dynamical calculations. On the other hand, it is known that friction contributes to dissipation of energy, which again naturally contributes to stability of the system. At the same time it can lead to control deviations. Regardless of the advantages and disadvantages, friction plays a significant role in physical situations and should therefore later be included in the model. 
%A side note is that this aspect is one of the challenges related to terrestrial snake robot locomotion as opposed to underwater locomotion.

%The second significant difference between dynamic HPFC and HPFC is the description of the constraints. In the HPFC method of West and Asada \cite{west1985method}, these are purely defined through the presence of closed kinematic chains. The closed kinematic chains are defined by including virtual joints connecting the robot and the environment. As the contact surface between the robot and the environment gets more complex, more virtual joints have to be introduced. Yoshikawa \cite{yoshikawa1987dynamic} instead describes the contact surface directly as a set of hypersurfaces. They are then differentiated and included in the dynamical calculations used for finding the control torques needed to realise a given desired trajectory. This generally makes the dynamic HPFC method more resilient. However, the individual constraints are not very complex in this project as all obstacles are simply modeled as static points in the world.
%When the research evolves, more advantage may be taken of this feature.

The traditional HPFC method is able to analyze whether or not a task is achievable with its use of essential task direction vectors and allowable position and force mappings. These essential directions have to be defined by the user. For dynamic HPFC, the allowable motion and force directions have to be pre-defined as well, but the user or planner algorithm additionally has to consider which task combinations are in fact achievable. For instance, the snake robot might be in a position to push against and shape itself along five obstacles, but depending on the composition of the snake robot and the distribution of these obstacles, achieving all control goals simultaneously might not be feasible. 

A possible solution is evaluating the mapping filters (\ref{eq:hpfc_fp}), (\ref{eq:hpfc_ff}) of West and Asada \cite{west1985method} that have zero rank if the task is outside the combined allowable and essential spaces. Ideally, the information gained from these filters should be exploited by the HOAL planner algorithm, leaving the controller to simply realise any task it is requested to.

\subsection{Mathematical formulations}

It should be mentioned that since it is the first time the dynamic HPFC method has been adapted to snake robots, the chosen formulations can still be challenged. The formulation of the contact points are based on the idea of which variables the HOAL algorithm will desire to control. That is the global orientation of the contact links and the directions along and against the obstacles. When the minimal CKC formulation was implemented and used to compute the Jacobians, these variables were no longer all defined with respect to the base frame, but rather the frame of the preceding contact point. The intuition behind what the desired variables should be was then somewhat lost. Still, this can easily be solved by defining all desired values with respect to the base frame and use homogeneous transformation matrices to input the right formats to the controller.

\subsection{Passive joints}\label{subsec:dis-passive-joints}

An issue that was encountered during the testing is that the dynamic HPFC method does not consider that some of the joints are passive and thus unactuated. Consequently it computes motor commands for these joints as well as for the active joints. These commands can not be realised and merely ignoring them leads to unwanted behaviours, as was seen in the experiment in \ref{sec:pas-pos}. This observation led to the suggested solution described in \ref{subsec:passive-joints}, where a subset of the joint variables ($\boldsymbol{\psi}$) are picked for reference following and the generalized control forces are calculated only for the active joints. This was also tested and presented in \ref{sec:pas-pos}, and led to a considerable improvement in the results.

It is however not believed that this is the most optimal solution to the problem, because it requires that the variable subset $\boldsymbol{\psi}$ is continuously determined and that the generalized coordinates and dynamics matrices are rearranged accordingly. There has so far not been developed any explicit method for determining the subset of the joint variables that are the most significant for achieving a given task. A lot of trial and error, both with the variable subset and tuning of the internal controllers, was conducted before satisfying results were attained in the experiment in \ref{sec:pas-pos}.

Furthermore, the arbitrariness term $(\mathbf{I}-  \mathbf{J}_t^+ \mathbf{J}_t)\mathbf{k}$ in (\ref{eq:dhpfc_qddd}) for the calculation of the desired joint accelerations $\ddot{\mathbf{q}}$ has not been researched extensively and thus not been included in the experiments. It is also more significant for snake robots with a large number of links where the arbitrariness of the motion is higher. One idea is however combining the choice of $\mathbf{k}$ with the choice of the subset $\boldsymbol{\psi}$ of the joint variables mentioned above. If the term $(\mathbf{I}-  \mathbf{J}_t^+ \mathbf{J}_t)\mathbf{k}$ can influence which joint variables are assigned the most vital values for reaching a goal, then $\boldsymbol{\psi}$ could be altered less frequently. This is simply a hypotheses, and other ways of exploiting the influence of $\mathbf{k}$ should be researched further as well. It should also be noted that since the vector $\mathbf{k}$ will be projected onto the nullspace of the robot, there are a lot of variations of $\mathbf{k}$ which will yield the same resulting projected vector. This decreases the span of variations that have to be investigated.

\subsection{Closed kinematic chains}

The analysis of the closed kinematic chains (CKCs) in \ref{subsec:task-restrictions} concludes that a higher number of joints leads to a higher level of controllability. Because the variables desired to control are related to the contact points, the arrangement of the obstacles in touch with the snake robot have an impact on the controllability as well. That is, if for instance all obstacles are gathered at the back part of the snake robot, it will not be able to exploit all of its joints to control the contact point variables. In other words, it is desired to have a snake robot with a large number of joints configured so that there are sufficiently many joints between every contact point to make control of the contacts close to independent of each other. A higher number of joints also allows for a higher number of solutions to every control problem and will thus increase the dexterity.

In case it is not important to control all contact points accurately, one should implement an algorithm that can find the largest CKC for every contact point desired to control that does not overlap with any other active CKCs. This way the utilization of the snake robot joints can be maximized and the controllability increased. On the other hand, it is sensible to always control all contact forces based on the fact that a mechanical snake robot can break or get damaged under too much pressure.

\subsection{Control structure}

When the complexity of a controller increases, it is typically dependent on a greater amount of feedback parameters from the system to execute all of its calculations. Thus, it is dependent on the accuracy of a greater amount of parameters. The dynamic hybrid position/force controller for the simulated snake robot has access to accurate data of position, velocity and force. These parameters are used both for the dynamic HPFC and for the inner control loops considering the desired forces and positions. Even though the data was considered to be of good quality, a great deal of tuning had to be conducted to optimize the inner control loops and approach an optimal controller. 

For this project, the velocities and displacements have been kept small in order to stay within the validity bounds of the mathematical model of the snake robot and its constraints, and the control parameters are tuned correspondingly. A more thorough tuning of control parameters should thus later be conducted for a faster, yet stable, response of the system.

The inner control loops, which consider the position and force errors, can be utilized to weigh the control of the different variables. That is, it is possible to control some variables more strictly than others if they are seen as more vital for achieving the propulsion goal. To figure out which variables are the most important, the specific necessary conditions for propulsion should be determined and analyzed.

Furthermore, PI control was chosen for the inner control loops since all motion was very slow either way. It is however suggested to implement PID or PD controllers in the continuation of the research of this project. In addition, the desired contact point accelerations were calculated directly from the desired position variables, rather than the velocities. It is not believed to have had any negative consequences in the scope of this project, but should also be improved in future implementations.

Singularity avoidance has not been included in the control for this project since singularities have not turned out to be prominent in the vicinity of the experiments. It should however be implemented when the scope of the testing is increased. If the assigned task does not pass through unavoidable singularities, it is in principle always possible to compute a joint trajectory along which the task Jacobian $\mathbf{J}_t$ is continuously full rank \cite{chiaverini2008kinematically}.

%Krevde mye mer tuning enn WA sin metode for å få tilfredsstillende resultater.


\section{Simulation limitations}

\subsection{The snake robot model}

The most important deficiency in the quality of the simulation experiments is that the number of links is quite low. The reason for this is explained in \ref{subsec:dis-diff}. In future experiments it is recommended to use a snake robot with a large number of links so that the controllability increases and artefacts or disturbances from control of adjacent contact points are minimal.

Another disadvantage is that the movements have to stay within the bounds of the model description, meaning contact with the contact links have to be maintained at all times. It was discovered that uncontrolled oscillations could occur when the snake robot violated this assumption. When the movement span increases it is also more likely that the snake robot ends up in or approaches a singular configuration, which leads to large joint velocities. However, the purpose of the experiments in this project is not achieving propulsion but rather test the dynamic HPFC method on a snake robot.

Another resulting limitation to the test quality is that the dynamical model is less significant and influential for very slow movements where the velocities are close to zero. Therefore, the dynamical part of the implemented method has not been tested as extensively as it could have been in a different test and model environment.

\subsection{Force sensor signal}

The force sensor signal from the physics simulator is way more unsteady than the other signals provided. A reason for this can be that calculating forces in collisions is very complex. The links in contact did not collide with the obstacles very frequently during simulations, however, it is believed that the unsteadiness of the signal is a result of the contact point on the contact link constantly changing. The displacements might not be big, but enough for the simulator to register a new contact point and thus recalculate the force with new parameters. It was observed that when the robot was lying completely still the force signals were much steadier. According to Guillaume \cite{GuillaumeSnakeSIM}, continuous contact is excluded because contact will only happen on the triangle edges of an object.

Despite the unsteady force signals, the controller is able to drive the contact forces to oscillate around the desired value. Lowpass filtering the force sensor signals was further observed to have a positive impact on the control, as can be seen in the experiment in \ref{sec:pos-force-control-exp}.

An unfortunate trait of the physics simulator is that it has a tendency to push the snake robot ever so slightly away from the obstacles when it is lying still. The movement is not significant, but the problem is that it results in the contact force signal and information about the contact point being lost.

\subsection{Simulator user-friendliness}

Using the ROS/Gazebo SnakeSIM simulator is not quite plug-n-play, and it is recommended to reserve enough time to understand the principles of the program structure and how it should be used. Both knowledge about the Ubuntu operating system and the use of the command line is required.

The previously made modules for this simulator are well developed in that it is intuitive which ones should be deployed. On the other hand, there is lacking documentation of the different modules and the simulator in general, which made familiarization with the program very time consuming.

\section{Dynamic HPFC for HOAL}

The experiment in \ref{sec:pos-force-control-exp} shows that simultaneous control of position and force against a single obstacle is achievable with the presented modifications to the dynamic HPFC method. This experiment is however not enough to determine whether or not dynamic HPFC is the best method to be used for HOAL. Further experiments with larger velocities, a higher number of snake robot links and simultaneous control of several contact points should be carried out to substantiate the idea. It can however be concluded that the results are an indication that the research is moving in the right direction.

Furthermore, it has been shown that the amount of links, and in particular the amount of links between contact points, is vital for the performance of the dynamic HPFC method and thus the HOAL. The results of the experiment in \ref{sec:2xminiJforce} suggest that at least two actuated joints are needed for accurate control of the force at one contact point. It is also achievable with one actuator, of course given that the snake robot is able to push against another obstacle to stay in place, but this is less optimal. Ideally, hyperredundant snake robots should be used, as described in \ref{subsec:task-restrictions}.

In order to focus the control on variables that are in fact desired to be controlled by a given HOAL algorithm, it was suggested in \ref{subsec:task-oriented} that the position and force filters from the traditional HPFC method can be combined with the dynamic HPFC method. This was proven to be successful in the experiment in \ref{sec:ess-pos}. However, several conditions and requirements have to be fulfilled for the proposed method to be valid, and it is thus not believed that it is scalable to bigger experiments with higher velocities. As mentioned in \ref{subsec:dis-diff}, it should still be researched if these filters can be exploited by any part of the HOAL algorithm.

%\subsection{Conditions for propulsion}