\chapter{\abstractname}

%Snake robots are robust serial link robots able to propel themselves through uneven and irregular terrain. As opposed to traditional mobile robots, the snake robot can utilize obstacles found in the environment to push itself forward along a predefined path. The aforementioned mode of locomotion is referred to as \textbf{obstacle aided locomotion (OAL)}. The challenge of both controlling the shape and contact forces of the snake robot has motivated the study of \textbf{hybrid position/force control (HPFC)}, with special emphasis on application within OAL.

Snake robots are serial link robots able to traverse a variety of different terrains. 
%Typical locomotion strategies for flat surfaces include lateral undulation and concertina locomotion. 
\textbf{Obstacle Aided Locomotion (OAL)} is a method for traversing terrains prone by obstacles like rocks and walls which a snake robot can utilize to push itself forward while following a predetermined path. In order to control the resulting obstacle contact forces and snake robot shape simultaneously, the \textbf{Hybrid Position/Force Control (HPFC)} method is studied. The combination of OAL and HPFC is referred to as \textbf{Hybrid Obstacle Aided Locomotion (HOAL)}.

In particular, this report is focused on the dynamic HPFC method, which integrates the robot dynamics into the control. The method is thoroughly studied and adapted to the snake robot case. Furthermore, some additions to the method are proposed and tested with the emphasis on considering the passive joints of the snake robot. 
The simulator SnakeSIM, which is based in the Robotic Operating System (ROS), has been used for all presented experiments. An explanation and evaluation of the simulation framework is provided in the report.

It has been found that the simultaneous control of force and position for a snake robot with several contact points is achievable, but highly dependent on the number of actuated joints relative to the number of contact points. More specifically, it was found that at least two actuated joints should be located between each contact point for successful control. Because the testing was limited to a snake robot with few links and very simple control goals, it is believed that this number might be higher for more complex scenarios and should be investigated further.



%Due to the complexity of the dynamical model for snake robots with a very large number of links, the experiments were limited to a six-link snake robot. For this reason, the 

\makeatletter
\ifthenelse{\pdf@strcmp{\languagename}{english}=0}


\makeatother



%TODO: Abstract in other language




