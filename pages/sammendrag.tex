\chapter{Sammendrag}

%Snake robots are serial link robots able to traverse a variety of different terrains. \textbf{Obstacle Aided Locomotion (OAL)} is a method for traversing terrains prone by obstacles like rocks and walls which a snake robot can utilize to push itself forward while following a predetermined path. In order to control the resulting obstacle contact forces and snake robot shape simultaneously, the \textbf{Hybrid Position/Force Control (HPFC)} method is studied. The combination of OAL and HPFC is referred to as \textbf{Hybrid Obstacle Aided Locomotion (HOAL)}.

%In particular, this report is focused on the dynamic HPFC method, which integrates the robot dynamics into the control. The method is thoroughly studied and adapted to the snake robot case. Furthermore, some additions to the method are proposed and tested. It has been found that the simultaneous control of force and position for a snake robot with several contact points is highly dependent on the composition of the snake robot, meaning its number of actuated joints relative to the number of contact points. 

%The ROS-based simulator, SnakeSIM, has been used for all presented experiments, and an explanation and evaluation of the simulation framework is provided in the report.

Slangeroboter er seriekoblede roboter som evner å traversere en mengde forskjellige terreng. \textbf{Obstacle Aided Locomotion (OAL)} eller hindringsbasert fremdrift er en metode for traversering av terreng preget av mye steiner eller liknende hindringer som slangeroboten kan utnytte for å dytte seg selv fram langs en forhåndsdefinert bane. For å kunne styre de resulterende kontaktkreftene fra hindringer og slangerobotens form samtidig er \textbf{Hybrid Position/Force Control (HPFC)} eller hybrid posisjons- og kraftstyring studert. Kombinasjonen av OAL og HPFC er omtalt som \textbf{Hybrid Obstacle Aided Locomotion (HOAL)}.

Denne rapporten fokuserer i hovedsak på den dynamiske HPFC-metoden der robotdynamikken er integrert i kontrollen. Metoden er studert grundig og nøye tilpasset slangerobottilfellet. Videre er noen forbedringer til kontrollmetoden foreslått og testet med ekstra vekt på passive ledd i slangeroboten.
Simulatoren SnakeSIM, som bygger på rammeverket Robotic Operating System (ROS), er tatt i bruk for utførelsen av alle eksperimenter. En forklaring og evaluering av simulatoren er gitt i rapporten.

Det ble bemerket at hybrid styring av kraft og posisjon for en slangerobot med flere kontaktpunkter er oppnåelig, men høyst avhengig av antallet slangerobotledd og kontaktpunkter. Eksperimentene antyder at minst to aktuerte ledd mellom hvert kontaktpunkt er nødvendig for tilfredsstillende kontroll. Fordi testingen var begrenset til en slangerobot med kun seks lenker og veldig enkle kontrollmål, så er det antatt at dette minstekravet kan være høyere for mer komplekse scenarioer.



\makeatletter
\ifthenelse{\pdf@strcmp{\languagename}{english}=0}


\makeatother



%TODO: Abstract in other language




